\DeclareTColorBox{paperbox}{O{} m}{%
	frame hidden,
	boxrule=0pt,
	enhanced,
	before skip=11pt plus 1pt,
	toptitle=3mm,
	boxsep=0.25ex,
	left=8pt,
	right=8pt,
	title={#2},
	arc=0mm,
	parbox=false,
	borderline north={1pt}{-0.5pt}{black},
	borderline south={1pt}{-0.5pt}{black},
	coltitle=black,
	fuzzy shadow={0mm}{-3.5pt}{-0.5pt}{0.4mm}{black!60!white},
	overlay={%
		\fill[black] (frame.south west) -- ++ (7pt,0) -- ++ (0,-5pt) -- cycle;
		\fill[black] (frame.north west) -- ++ (7pt,0) -- ++ (0,5pt) -- cycle;
		\fill[black] (frame.north east) -- ++ (-7pt,0) -- ++ (0,5pt) -- cycle;
		\fill[black] (frame.south east) -- ++ (-7pt,0) -- ++ (0,-5pt) -- cycle;
		},
	after={\vspace{10pt plus 1pt}\noindent},
	#1
}
\lstdefinestyle{PythonStyle1}{ % Define a style for your code snippet, multiple definitions can be made if, for example, you wish to insert multiple code snippets using different programming languages into one document
	language=Python, % Detects keywords, comments, strings, functions, etc for the language specified
	backgroundcolor=\color[RGB]{255,251,204}, % Set the background color for the snippet - useful for highlighting
	basicstyle=\footnotesize\ttfamily, % The default font size and style of the code
	breakatwhitespace=false, % If true, only allows line breaks at white space
	breaklines=true, % Automatic line breaking (prevents code from protruding outside the box)
	captionpos=b, % Sets the caption position: b for bottom; t for top
	commentstyle=\usefont{T1}{pcr}{m}{sl}\color[rgb]{0.0,0.4,0.0}, % Style of comments within the code - dark green courier font
	deletekeywords={}, % If you want to delete any keywords from the current language separate them by commas
	%escapeinside={\%}, % This allows you to escape to LaTeX using the character in the bracket
	firstnumber=1, % Line numbers begin at line 1
	frame=single, % Frame around the code box, value can be: none, leftline, topline, bottomline, lines, single, shadowbox
	frameround=tttt, % Rounds the corners of the frame for the top left, top right, bottom left and bottom right positions
	keywordstyle=\color{Blue}\bfseries, % Functions are bold and blue
	morekeywords={}, % Add any functions no included by default here separated by commas
	numbers=left, % Location of line numbers, can take the values of: none, left, right
	numbersep=10pt, % Distance of line numbers from the code box
	numberstyle=\tiny\color{Gray}, % Style used for line numbers
	rulecolor=\color{black}, % Frame border color
	showstringspaces=false, % Don't put marks in string spaces
	showtabs=false, % Display tabs in the code as lines
	stepnumber=5, % The step distance between line numbers, i.e. how often will lines be numbered
	stringstyle=\color{Purple}, % Strings are purple
	tabsize=2, % Number of spaces per tab in the code
}
\lstdefinestyle{customStylePythonDark}{
    language=Python,
    numbers=left,%position of line numbers (left/right/none, i.e. no line numbers)
    basicstyle=\footnotesize\ttfamily\color[RGB]{255,255,255},%font size/family/etc. for source (e.g. basicstyle=\ttfamily\small)
    numberstyle=\color[RGB]{0,0,0},%style used for line-numbers
    backgroundcolor=\color[RGB]{33,36,33},%colour for the background. External color or xcolor package needed.
    commentstyle=\itshape\color[RGB]{153,153,153},%style of comments in source language.
    keywordstyle=\bfseries\color[RGB]{143,217,68},%style of keywords in source language (e.g. keywordstyle=\color{red})
    identifierstyle=\color[RGB]{101,197,222},
    stringstyle=\color[RGB]{236,118,0},%style of strings in source language
    belowcaptionskip=1\baselineskip,%is the vertical space respectively above or below each caption
    breaklines=true,%automatic line-breaking
    frame=single,%showing frame outside code (none/leftline/topline/bottomline/lines/single/shadowbox)
    xleftmargin=\parindent,
    showstringspaces=false,
    captionpos=b,%position of caption (t/b)
    showspaces=false,%emphasize spaces in code (true/false)
    showtabs=false,%emphasize tabulators in code (true/false)
    tabsize=4,%default tabsize
    lineskip=0.5em,
    usekeywordsintag=false,
    postbreak=\raisebox{0ex}[0ex][0ex]{\ensuremath{\color{VioletBlue}\hookrightarrow\space}}
}

\lstdefinestyle{customStylePythonLight}{
    language=Python,
    numbers=left,%position of line numbers (left/right/none, i.e. no line numbers)
    basicstyle=\footnotesize\ttfamily\color[RGB]{0,0,0},%font size/family/etc. for source (e.g. basicstyle=\ttfamily\small)
    numberstyle=\color[RGB]{0,0,0},%style used for line-numbers
    backgroundcolor=\color[RGB]{255,255,255},%colour for the background. External color or xcolor package needed.
    commentstyle=\itshape\color[RGB]{52,121,54},%style of comments in source language.
    keywordstyle=\bfseries\color[RGB]{138,18,130},%style of keywords in source language (e.g. keywordstyle=\color{red})
    stringstyle=\color[RGB]{0,0,238},%style of strings in source language
    belowcaptionskip=1\baselineskip,%is the vertical space respectively above or below each caption
    breaklines=true,%automatic line-breaking
    frame=shadowbox,%showing frame outside code (none/leftline/topline/bottomline/lines/single/shadowbox)
    rulesepcolor=\color{black},
    xleftmargin=\parindent,
    showstringspaces=false,
    captionpos=b,%position of caption (t/b)
    showspaces=false,%emphasize spaces in code (true/false)
    showtabs=false,%emphasize tabulators in code (true/false)
    tabsize=4,%default tabsize
    lineskip=0.5em,
    usekeywordsintag=false,
    postbreak=\raisebox{0ex}[0ex][0ex]{\ensuremath{\color{VioletBlue}\hookrightarrow\space}}
}
