\chapter{数据结构:小海龟赛跑}
数据结构是一个比较抽象的概念,我们会更多地介绍例子而不是概念。
\subsection{Python基本数据结构}
\begin{paperbox}{\textbf{Learning By Reading}\starthree}
浏览\href{https://www.jianshu.com/p/75425f405c25}{材料},对数据结构有大致了解。
简单概括起来,数据结构让数据“有结构”,更方便存储、访问、修改等操作。
\end{paperbox}
如果我们有两个小海龟,我们可以分别叫它们$t1$,$t2$,
那如果有10个,100个呢?\\
这里,小海龟可以看作是数据,我们需要数据结构来让我们更方便地操纵这些小海龟。\\
$color$方法可以改变小海龟的颜色(试着用$help$了解$color$的用法),
如果每个小海龟的颜色不同,又要如何管理这些颜色呢?\\
\begin{paperbox}{\textbf{Learning By Reading}\starfive}
阅读\href{http://blog.miskcoo.com/2016/07/python-fundamental-data-structures}{材料}
,了解Python的基本数据结构,并为刚才的两个问题选择合适的数据结构。
\end{paperbox}
刚才的材料中提到“列表的迭代”,其中用了$for$循环。
$for$循环和$while$循环有相似之处,区别在于$for$用于遍历“可迭代”的对象
(list,dict,tuple,str都是可迭代的对象),
$while$用于条件判断。
\subsection{迭代器}
\begin{paperbox}{\textbf{Learning By Reading}\starfive}
阅读\href{http://wiki.jikexueyuan.com/project/explore-python/Advanced-Features/iterator.html}{材料}(的前半部分),
了解迭代器与$for$。
\end{paperbox}
\begin{paperbox}{\textbf{Learning By Reading}\starfive}
生成list有另外一种强大的方式“列表解析”。看\href{http://codingpy.com/article/python-list-comprehensions-explained-visually/}{材料}。\\
比如,想要一个10个小海龟的list,我们可以用$[turtle.Turtle() for i in range(10)]$来生成。
\end{paperbox}
\begin{paperbox}{\textbf{Learning By Doing}\starthree}
小海龟赛跑:将若干个不同颜色的小海龟排成一排,
轮流随机向前爬一定距离,最先爬过终点的小海龟获胜。
效果示例见\url{https://codeclubprojects.org/en-GB/python/turtle-race/}。
\end{paperbox}
\subsection{关键字参数}
你可能已经注意到,$print$函数可以带任意多个参数,如$print(1, 2, 3)$,这是如何做到的?
\begin{paperbox}{\textbf{Learning By Reading}\starfive}
阅读\href{http://www.cnblogs.com/bingabcd/p/6671368.html}{材料},
了解Python函数的参数。
\end{paperbox}
