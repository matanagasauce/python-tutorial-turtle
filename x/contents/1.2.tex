\chapter{计算机如何工作}
本章我们将对计算机如何工作有一个初步的认识。\\
大体上来说,为了让一台比较现代的计算机运行,我们需要能完成以下功能的组件:\\
\begin{itemize}
\item[-]执行指令
\item[-]存储指令、数据
\item[-]知道下一条指令是什么
\item[-]显示图像、声音
\end{itemize}
\section{CPU与内存}
CPU和内存都是现代计算机中不可缺少的组成部分。\\
一个典型的CPU有算术逻辑单元,寄存器和控制单元,算术逻辑单元(ALU)用来执行具体的指令,比如基本的数学计算(加减乘除等)、读写数据;
“知道下一条指令是什么”由控制单元(Control unit)完成(CU还有其他功能),大多数时候指令是按照从前到后的顺序执行的,
但有时候,我们会希望突然跳到某条指令上。
寄存器(Register)用来存储,它们数量比较少,所以每个都有自己的名字,在x86中,它们叫$AX$,$BX$,$\cdots$在ARM中,它们叫$R0$,$R1$,$\cdots$
\footnote{\url{https://en.wikipedia.org/wiki/List_of_CPU_architectures}}。
不同厂商不同型号的CPU在设计和实现上差异较大。目前最常见的CPU架构是ARM和x86。
\begin{paperbox}{\textbf{Learning By Reading}\starone}
ARM是一种指令集架构(Instruction set architexture),大多数智能手机用的都是这种架构。
ARM指令集是精简指令集(RISC),意思是ARM在设计上通过减少单条指令的工作量来减少单条指令执行的时间(CPI)
\footnote{\url{https://en.wikipedia.org/wiki/Reduced_instruction_set_computer}}。
浏览\href{http://infocenter.arm.com/help/topic/com.arm.doc.qrc0006ec/QRC0006_UAL16.pdf}{材料},
了解ARM指令集都可以做哪些事情。同时,也可以在维基百科中查询相应词条,了解更多关于ARM、RISC的信息。
\end{paperbox}
\begin{paperbox}{\textbf{Learning By Reading}\starone}
x86是另一种指令集架构,大多数电脑用的是这种架构。
阅读以下材料,了解x86指令集有哪些指令。\\
% \url{https://en.wikipedia.org/wiki/X86_instruction_listings}\\
% \url{https://zh.wikibooks.org/zh-cn/X86組合語言/基本指令集}
\url{https://www.tutorialspoint.com/microprocessor/microprocessor_8086_instruction_sets.htm}
\end{paperbox}
\begin{paperbox}{\textbf{Learning By Reading}\startwo}
阅读材料,了解ARM和x86之间的优劣比较。\\
\href{http://ihyperwin.iteye.com/blog/1701132}{ARM与x86 CPU架构对比区别}\\
\href{https://www.zhihu.com/question/19846434}{ARM、x86的比较}\\
\href{https://hellolynn.hpd.io/2017/04/14/看arm如何搶走x86市場?英特爾被逆襲下的策略/}{ARM、x86之争}
\href{http://www.cnblogs.com/croot/archive/2012/11/24/3235140.html}{ARM指令集和x86指令集的比较}
\end{paperbox}
内存是一种存储设备,用来存储数据和指令供CPU使用,完成人类大脑的“记忆”功能。
比如要计算$5+6$,计算机要存储:\\
\begin{itemize}
\item[-]数据
\begin{itemize}
\item[-]数字5,存放在地址A
\item[-]数字6,存放在地址B
\end{itemize}
\item[-]指令
\begin{itemize}
\item[-]指令“把地址A中存放的数字和地址B中存放的数字相加并存放在地址C中
\footnote{这里只是举例,实际上需要用到更多的指令来完成这个任务。}
\footnote{需要注意的是,计算出来的结果也需要存储在某个地方。人类
用大脑进行计算的时候,其实也会存储(记住)这个结果,只是我们一般不去注意。}
”
\end{itemize}
\end{itemize}
数据以多种多样的形式存在,它们在存储器中存储的方式可能也不同。
\begin{paperbox}{\textbf{Learning By Reading}\starone}
阅读以下材料,了解\href{https://www.crucial.cn/support/faq/the-role-of-memory-in-the-computer}{内存有什么用},
\href{http://www.cppblog.com/prayer/archive/2009/08/17/93594.html}{内存中都有什么}。
\end{paperbox}
\begin{paperbox}{\textbf{Learning By Thinking}\startwo}
查阅ARM、x86指令集以及助教,说一说$5+6$需要用到哪些指令,$1+\cdots+10$呢?
\end{paperbox}
衡量包括内存在内的存储器有两个重要的指标:速度和大小
\footnote{\url{https://en.wikipedia.org/wiki/Memory_hierarchy}}
。速度越快价格越高,大小越大价格也越高,这是显而易见的。
出于价格的考虑,两者之间需要权衡。
现代计算机中通常有不同速度的存储器,包括寄存器(Register)、高速缓存(Cache)、随机存取存储器(Random-access memory,RAM)。
从前到后越来越慢、越来越大。
\begin{paperbox}{\textbf{Learning By Reading}\starthree}
阅读材料,了解\href{http://blog.csdn.net/hellojoy/article/details/54744231}{内存、cache和寄存器之间的关系及区别}
\end{paperbox}
当需要数据的时候,计算机会先去高速的存储器中寻找,找不到的时候再去低速的存储器中寻找,
并把这部分数据载入到高速的存储器中。在很多情况下,一段数据被使用之后在未来短时间内会再次被使用,因此,
这种方法通常能节省大量时间。而这种特性叫做“时间局部性(Temporal locality)”。
除此之外,还有空间局部性(Spatial locality)等局部性原理(Principle of locality)保证代码高效运行。
\begin{paperbox}{\textbf{Learning By Reading}\startwo}
阅读材料,了解空间局部性、时间局部性:\\
\url{http://blog.csdn.net/qq_33083519/article/details/55106860}\\
\url{http://www.cnblogs.com/yanlingyin/archive/2012/02/11/2347116.html}\\
分支局部性:\\
\url{https://stackoverflow.com/questions/11227809/why-is-it-faster-to-process-a-sorted-array-than-an-unsorted-array/11227902#11227902}\\
里面牵涉到具体的代码现在可能看不懂,但学习完之后请回头再来看这部分。
\end{paperbox}
\section{外部设备}
外部设备(Peripheral)是用来进行输入输出的辅助设备
\footnote{\url{https://en.wikipedia.org/wiki/Peripheral}}。
输入指的是让计算机获取一些信息,输出指的是将计算机中存储的信息呈现出来。
这里的“输入输出”可以是任何形式的输入输出,比如,鼠标、键盘、触摸板、麦克风都是输入设备,
而显示器、音箱、打印机都是输出设备。\\
% 计算机通过“中断(Interrupt)
% \footnote{\url{https://en.wikipedia.org/wiki/Interrupt}}
% ”的方式与外部设备交互。\\
% 比如,一台x86计算机想要将显示器上的某个像素点改成蓝色。
% 首先,计算机执行$INT 10H$
\begin{paperbox}{\textbf{Learning By Reading}\starone}
阅读以下材料,进一步了解外部设备如何工作:\\
\href{https://www.zhihu.com/question/20722310}{计算机底层如何访问显卡}\\
\href{https://www.zhihu.com/question/39846396}{计算机如何调用硬件}
\end{paperbox}
\section{编译与解释}
高级语言需要转化为机器语言才能被执行,方式有两种,编译和解释,简单来说,它们的不同在转化的时间上,
编译选择在程序运行之前转化,而解释转化一句执行一句。
Python是解释型语言。
\begin{paperbox}{\textbf{Learning By Reading}\starthree}
阅读以下材料:\\
\href{https://www.zhihu.com/question/21486706/answer/32975999}{编译和解释的一个形象的比喻}\\
\href{http://alexyyek.github.io/2015/01/08/CompileAndInterpreter/}{更详细地介绍编译和解释}
\end{paperbox}
\begin{paperbox}{\textbf{Learning By Thinking}\starthree}
之前我们提到,有时候用汇编语言可以提高程序运行的效率,为什么呢?
\end{paperbox}
抽象地去解释的话,这个问题其实很好理解。做一件事通常有很多种方法,
同样地,不一样的两段代码可能实现的是同一个功能,但编译器/解释器只能选择其中一种,
很多时候,被选择的不一定是最好的。
