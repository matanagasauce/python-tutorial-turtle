\part{Python编程}
前面做了好多准备工作,终于开始正题了$\cdots$
在这份教程中,我们将介绍Python的基本语法,并围绕“小海龟画图”提供丰富的例子和练习。\\
“小海龟画图”是一个兼有实用与趣味的模块(什么叫模块?教程中马上就会提到。),我们也提供了若干例子,
读者可以先尝试运行这些例子,领略“小海龟画图”的神奇。\\
相信,在学完本教程之后,你可以写出更有创意的作品。
\chapter{库与函数调用:前进与转弯}
在本章中,我们将学会控制小海龟前进、转弯,并画出一个正方形。
\section{函数}
\subsection{什么是函数}
数学中有函数的概念,如二次函数$f(x) = x^2$将$x$映射到$x^2$。
一个Python中的函数也可以实现类似功能,输入一个数,输出这个数的平方。
Python函数也能完成数学中的函数做不到的功能,比如在显示器上画一个正方形。
一般来说,实现某种特定功能、可以重复使用的一段代码就是一个函数
\footnote{\url{https://www.tutorialspoint.com/python/python_functions.htm}}
。在不同场合,函数也会被叫做其他名字,比如子程序、过程、方法等
\footnote{\url{https://en.wikipedia.org/wiki/Subroutine}}
。\\
\subsection{一些名词解释}
数学证明中经常有“设$f(x)=\cdots$”,这里的“设”在Python中叫做“\textbf{定义}”,即“定义一个函数”。
函数的使用叫“\textbf{调用}”。\\
刚才的例子$f(x) = x^2$是一个一元函数,这里的“元”在Python中叫做“\textbf{参数}”,即“函数$f(x)$带有一个参数”。
函数也可以没有参数。
\section{模块}
\subsection{什么是模块}
一个Python文件就是一个模块。一个模块通常包含一系列函数,让我们不必从零开始编写代码
\footnote{\url{https://www.liaoxuefeng.com/wiki/0014316089557264a6b348958f449949df42a6d3a2e542c000/0014318447437605e90206e261744c08630a836851f5183000}}
。
\subsection{在Python中使用模块和函数}
模块需要有自己的名字。在Python中,用
\begin{lstlisting}[style=PythonStyle1]
	import xxx
\end{lstlisting}
告诉解释器我们要使用模块$xxx$,用
\begin{lstlisting}[style=PythonStyle1]
	xxx.f(1, 2)
\end{lstlisting}
来调用$xxx$模块中的函数$f$,两个参数分别是$1$和$2$。
\begin{paperbox}{\textbf{Learning By Doing}\starfive}
输入以下代码,保存成文件并运行,观看结果。
\begin{lstlisting}[style=PythonStyle1, caption=Rectangle]
import turtle
turtle.fd(50)
turtle.rt(90)
turtle.fd(50)
turtle.rt(90)
turtle.fd(50)
turtle.rt(90)
turtle.fd(50)
turtle.rt(90)
\end{lstlisting}
\end{paperbox}
$turtle$便是本教程中作为核心介绍的模块,这个模块包含了许多控制小海龟画图的函数。
比如$turtle.fd$让小海龟前进($fd$是$forward$的缩写),带一个参数,表示前进的距离;
$turtle.rt$让小海龟右转($rt$是$right$的缩写),带一个参数,表示转动的角度。\\
类似,$turtle.bk$和$turtle.lt$让小海龟后退、左转。
\begin{paperbox}{\textbf{Learning By Doing}\starfive}
让小海龟倒退着画一个正五边形。
\end{paperbox}
\begin{paperbox}{\textbf{Learning By Thinking\starthree}}
数学函数与Python中的函数有什么异同?
\end{paperbox}
数学中的函数是一个映射,而计算机函数则不一定。
甚至,计算机函数可以用来产生随机数,这样的函数显然不是数学函数。
\section{关键字参数}

