\chapter{定义类;迷宫}
和函数一样,Python中我们也可以定义自己的类。
\subsection{定义}
\begin{paperbox}{\textbf{Learning By Reading}\starfive}
阅读\href{https://www.liaoxuefeng.com/wiki/0014316089557264a6b348958f449949df42a6d3a2e542c000/001431864715651c99511036d884cf1b399e65ae0d27f7e000}{材料}
,了解如何定义类和类中的属性、方法。\\
\end{paperbox}
也就是说,如果有
\begin{lstlisting}[style=PythonStyle1, caption=Rectangle]
def function_1(self, x):
    return self.x + x

class A(object):
    method_1 = function_1

a = A()
a.x = 2
\end{lstlisting}
这时候,$a.method_1(2)$、$function_1(a, 2)$、$A.method_1(a, 2)$三者是等价的。
\subsection{装饰器}
\begin{paperbox}{\textbf{Learning By Reading}\starfive}
阅读\href{http://www.cnblogs.com/Jerry-Chou/archive/2012/05/23/python-decorator-explain.html}{材料},了解装饰器。\\
\end{paperbox}
简单来说,
\begin{lstlisting}[style=PythonStyle1, caption=Rectangle]
@deco
def func(x):
    pass
\end{lstlisting}
等价于
\begin{lstlisting}[style=PythonStyle1, caption=Rectangle]
def func(x):
    pass
func = deco(func)
\end{lstlisting}
\subsection{类方法}
\begin{paperbox}{\textbf{Learning By Reading}\starfive}
阅读材料,了解\href{http://blog.csdn.net/handsomekang/article/details/9615239}{类方法、静态方法}和
\href{http://python.jobbole.com/80955/}{属性函数}。
\end{paperbox}
\begin{paperbox}{\textbf{Learning By Thinking}\starfive}
$classmethod$、$staticmethod$、$property$都是装饰器,
想想它们可以怎样实现。
\end{paperbox}
\begin{paperbox}{\textbf{Learning By Doing}\startwo}
试着实现$my_classmethod$、
$my_staticmethod$、$my_property$。
\end{paperbox}
\begin{paperbox}{\textbf{Learning By Doing}\starfive}
阶段性测试:画一个随机生成的迷宫,把小海龟放在里面,让它自己走出来。\\
迷宫的生成参考\href{http://maskray.me/blog/2012-11-02-perfect-maze-generation}{这里}。
\end{paperbox}
