\chapter{运算符、while循环:多边形}
\subsection{数}
如前所述,Python中的数字也是对象。
一个数的方法有加、减、乘、除等等。\\
如果我们要画一个正五边形,我们需要知道正五边形
的每个角多少度。一个方法是预先算好,
另一个方法是让Python帮我们算。
\begin{paperbox}{\textbf{Learning By Doing}\starfive}
运行以下代码,观察结果。
\begin{lstlisting}[style=PythonStyle1, caption=Rectangle]
x = int(360).__div__(5)
t1.fd(50)
t1.rt(x)
t1.fd(50)
t1.rt(x)
t1.fd(50)
t1.rt(x)
t1.fd(50)
t1.rt(x)
t1.fd(50)
t1.rt(x)
print(type(x))
\end{lstlisting}
\end{paperbox}
这段代码中,我们首先创建了一个$int$(整数)类型的对象,参数$360$
表示这个整数的值是360。
$__div__$是创建出来的这个对象的一个方法,作用是做除法,返回除法的结果并且
再赋值给$x$。这样,我们就得到了一个五边形的外角度数。\\
这段代码看起来很蠢,所以Python为我们提供了便利,写成$x = 360 / 5$就好。\\
最后我们发现,$x$的类型是一个浮点型$float$。
\begin{paperbox}{\textbf{Learning By Reading}\starfive}
阅读材料,了解Python的\href{http://www.runoob.com/python3/python3-basic-operators.html}{运算符}
以及运算符对应的方法:
\href{https://segmentfault.com/a/1190000007256392#articleHeader7}{运算符相关的魔术方法}。
\end{paperbox}
\subsection{条件与循环}
画五边形的时候,我们把$t1.fd(50)$和$t1.rt(x)$写了五遍,为了不做重复的机械劳动,
我们有了循环。\\
\begin{paperbox}{\textbf{Learning By Doing}\starfive}
运行以下代码,观察结果。
\begin{lstlisting}[style=PythonStyle1, caption=Rectangle]
iter = 0
while iter != 5:
    t1.fd(50)
    t1.rt(x)
    iter += 1
\end{lstlisting}
\end{paperbox}
这段话也画了一个五边形。
\begin{paperbox}{\textbf{Learning By Reading}\starfive}
阅读材料,了解\href{http://blog.csdn.net/leexide/article/details/17359943}{条件语句}和
\href{http://www.runoob.com/python3/python3-loop.html}{while循环}的语法和语义。
\end{paperbox}
\subsection{关键字}
保留字(英语:Reserved word),有时也叫关键字(keyword),是编程语言中的一类语法结构。 在特定的编程语言里,这些保留字具有较为特殊的意义,并且在语言的格式说明里被预先定义。
\footnote{\url{https://zh.wikipedia.org/zh-hans/保留字}}\\
之前接触到的$import$也是关键字。
\subsection{定义函数}
之前的章节中我们使用了一些Python已有的函数,
这些函数不一定有我们需要的功能,比如,一个画正多边形的函数
看起来会很好用,但是turtle库中并没有直接提供。
所以我们需要自己定义函数。
\begin{paperbox}{\textbf{Learning By Reading}\starfive}
阅读材料,了解定义函数和匿名函数的语法。\\
\href{http://www.runoob.com/python3/python3-function.html}{Python3 函数:定义一个函数}\\
\href{https://www.cnblogs.com/evening/archive/2012/03/29/2423554.html}{lambda 介绍}
\end{paperbox}
\begin{paperbox}{\textbf{Learning By Doing}\starfive}
定义函数$polygon(n, x)$,功能为画一个边长为$x$的正$n$边形,并用该函数
画一个正方形。
\end{paperbox}
\begin{paperbox}{\textbf{Learning By Doing}\starfour}
画一个长\href{https://imgsa.baidu.com/exp/w=480/sign=0305f63ebf8f8c54e3d3c4270a282dee/d0c8a786c9177f3ecb5dccdd7bcf3bc79e3d56f0.jpg}{这样}的房子。
\end{paperbox}
函数中可以调用其他函数,也可以调用自己,即递归。\\
通过递归可以画出一些很炫的分形(Fractal)图案。
\begin{paperbox}{\textbf{Learning By Doing}\starthree}
\href{http://www.matrix67.com/blog/archives/6231}{这里}有些分形图形的例子,demo中的fractal.py中也有几个
函数。选择自己喜欢的图形,试着画出来。\\
提示:画这样图形的要领是将复杂的图形分解为自身的重复,对自身的重复即可以用递归实现。
\end{paperbox}
