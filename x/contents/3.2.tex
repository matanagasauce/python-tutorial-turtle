\chapter{运算符、while循环:多边形}
\subsection{数}
如前所述,Python中的数字也是对象。
一个数的方法有加、减、乘、除等等。\\
如果我们要画一个正五边形,我们需要知道正五边形
的每个角多少度。一个方法是预先算好,
另一个方法是让Python帮我们算。
\begin{paperbox}{\textbf{Learning By Doing}\starfive}
运行以下代码,观察结果。
\begin{lstlisting}[style=PythonStyle1, caption=Rectangle]
x = int(360).__div__(5)
t1.fd(50)
t1.rt(x)
t1.fd(50)
t1.rt(x)
t1.fd(50)
t1.rt(x)
t1.fd(50)
t1.rt(x)
t1.fd(50)
t1.rt(x)
print(type(x))
\end{lstlisting}
\end{paperbox}
这段代码中,我们首先创建了一个$int$(整数)类型的对象,参数$360$
表示这个整数的值是360。
$__div__$是创建出来的这个对象的一个方法,作用是做除法,返回除法的结果并且
再赋值给$x$。这样,我们就得到了一个五边形的外角度数。\\
这段代码看起来很蠢,所以Python为我们提供了便利,写成$x = 360 / 5$就好。\\
最后我们发现,$x$的类型是一个浮点型$float$。
\begin{paperbox}{\textbf{Learning By Reading}\starfive}
阅读材料,了解Python的\href{http://www.runoob.com/python3/python3-basic-operators.html}{运算符}
以及运算符对应的方法:
\href{https://segmentfault.com/a/1190000007256392#articleHeader7}{运算符相关的魔术方法}。
\end{paperbox}
\subsection{循环}
画五边形的时候,我们把$t1.fd(50)$和$t1.rt(x)$写了五遍,为了不做重复的机械劳动,
我们有了循环。\\
\begin{paperbox}{\textbf{Learning By Doing}\starfive}
运行以下代码,观察结果。
\begin{lstlisting}[style=PythonStyle1, caption=Rectangle]
iter = 0
while iter != 5:
    t1.fd(50)
    t1.rt(x)
    iter += 1
\end{lstlisting}
\end{paperbox}
这段话也画了一个五边形。
\begin{paperbox}{\textbf{Learning By Reading}\starfive}
阅读\href{http://www.runoob.com/python3/python3-loop.html}{材料},了解while循环的语法和语义。
\end{paperbox}
\begin{paperbox}{\textbf{Learning By Doing}\starfive}
画一个正36边形。思考为什么正36边形长这样。
\end{paperbox}
\begin{paperbox}{\textbf{Learning By Doing}\starfive}
画一个长\href{https://imgsa.baidu.com/exp/w=480/sign=0305f63ebf8f8c54e3d3c4270a282dee/d0c8a786c9177f3ecb5dccdd7bcf3bc79e3d56f0.jpg}{这样}的房子。
\end{paperbox}
