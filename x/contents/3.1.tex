\chapter{类、方法、调用:前进与转弯}
Python是一门面向对象语言,在Python中有“一切皆对象”的说法。\\
在本章中,我们将学会创造一个小海龟对象,并控制它前进、转弯,画出一个正方形。
\section{模块}
\subsection{什么是模块}
之前的章节里面我们看到计算机可以直接执行的指令比较有限,我们编程的时候显然不可能每次都从头开始。
一个Python文件就是一个模块。
\footnote{\url{https://www.liaoxuefeng.com/wiki/0014316089557264a6b348958f449949df42a6d3a2e542c000/0014318447437605e90206e261744c08630a836851f5183000}}
模块中有一些预先写好的代码,让我们可以方便的使用。\\
\subsection{在Python中使用模块}
模块需要有自己的名字。在Python中,用
\begin{lstlisting}[style=PythonStyle1]
	import xxx
\end{lstlisting}
告诉解释器我们要使用模块$xxx$
\section{对象}
\subsection{什么是对象}
对象(Object)是来源于现实世界中“对象”的概念。
\begin{paperbox}{\textbf{Learning By Doing}\starfive}
阅读\href{http://blog.csdn.net/linyixiao88/article/details/50833502}{材料}的前几段(出现代码之前的段落),
了解对象、类、属性、方法。
\end{paperbox}
以小海龟画图为例,小海龟便是对象。小海龟的颜色、形状、爬行速度等是它的属性。
小海龟的方法有前进、转弯等等。\\
在编程的过程中,我们可以把自己想象成这个虚拟世界的神明,我们要创造出小海龟们,
并给它们下达指令。\\
\subsection{如何创造对象}
\begin{paperbox}{\textbf{Learning By Doing}\starfive}
运行以下代码,观察结果。
\begin{lstlisting}[style=PythonStyle1, caption=Rectangle]
import turtle
t1 = turtle.Turtle()
t2 = turtle.Turtle()
\end{lstlisting}
\end{paperbox}
以上代码中,$import turtle$告诉解释器我们需要使用小海龟画图的模块,
$t1 = turtle.Turtle()$创建了一个$Turtle$(小海龟)对象,名字是$t1$。
$t2 = turtle.Turtle()$创建了一个$Turtle$对象,名字是$t2$。
可以看到在屏幕正中央出现了一只三角形,不要怀疑,那就是我们的小海龟。\\
为什么创建了两只却只看到了一只?因为重合在一起了,下一节我们将移动小海龟,这样就能看见它们了。
\section{方法、函数}
\subsection{什么是方法}
方法(Method)是实现某种特定功能、可以重复使用的一段代码
\footnote{\url{https://www.tutorialspoint.com/python/python_functions.htm}}
。在不同场合,这样的一段代码也会被叫做其他名字,比如子程序、过程、函数等
\footnote{\url{https://en.wikipedia.org/wiki/Subroutine}}
。这段代码出现在一个类中,通常叫\textbf{方法}。
如果是不依赖一个类的“方法”,一般叫\textbf{函数}。\\
数学中有函数的概念,如二次函数$f(x) = x^2$将$x$映射到$x^2$。
一个Python中的方法也可以实现类似功能,比如一个正方形对象,求面积是它的一个方法。(相应地,面积可以是它的一个属性。)
Python中的方法也能完成数学中的函数做不到的功能,比如对于小海龟对象,画一个正方形也是它的一个方法。\\
刚才的例子$f(x) = x^2$是一个一元函数,这里的“元”在Python中叫做“\textbf{参数}”,即“$f(x)$带有一个参数”。
方法也可以没有参数。\\
\begin{paperbox}{\textbf{Learning By Doing}\starfive}
运行以下代码,观察结果。
\begin{lstlisting}[style=PythonStyle1, caption=Rectangle]
import turtle
t1 = turtle.Turtle('turtle')
t2 = turtle.Turtle('turtle')
t1.lt(90)
t1.fd(50)
\end{lstlisting}
\end{paperbox}
可以看到,屏幕上一只小海龟动了起来,而另一只小海龟停留在原地。\\
并且,通过在创建对象的时候带上参数`turtle',我们让小海龟真的变成了海龟的形状。\\
$t1.fd$让小海龟$t1$前进($fd$是$forward$的缩写),这个方法带一个参数,表示前进的距离;
$t1.lt$让小海龟$t1$右转($lt$是$left$的缩写),带一个参数,表示转动的角度。\\
类似,$t1.bk$和$t1.rt$让小海龟后退、左转。
\begin{paperbox}{\textbf{Learning By Doing}\starfive}
让小海龟$t2$画一个正五边形。
\end{paperbox}
\begin{paperbox}{\textbf{Learning By Thinking\starthree}}
数学函数与Python中的函数有什么异同?
\end{paperbox}
数学中的函数是一个映射,计算机函数的概念从数学函数中产生,可以看作是从参数定义域到代码的映射。
然而,在编程中相比于函数的代码,我们更关心这段代码的运行结果。从参数的定义域到运行结果却并不一定是一个映射。
比如,计算机函数可以用来产生随机数,这显然不是数学函数。
\section{Python:一切皆对象}
刚才的代码里,总共出现了多少对象?\\
两个小海龟…除此之外呢?\\
Python是一门面向对象语言,并且有说法号称“一切皆对象”。
一个模块$turtle$也是对象,一个类$turtle.Turtle$也是对象,
一个方法$t1.lt$也是对象,一个数字$50$也是对象。\\
\section{一些有用的工具}
Python有一些“内建函数”,指的是不需要任何模块就可以使用的函数。
我们先行介绍几个很有用的内建函数。\\
$print$,它可以在控制台打印(输出)信息。\\
比如一个对象的某属性,我们希望知道这个属性的值,就可以用$print$打印出来。\\
$type$函数返回一个对象的类型,即这个对象“属于”哪个类。
$help$函数输出(即$print$)一个对象的所有方法、属性,以及对应的文档
(文档一般是开发者写的,不写文档是被允许的,所以也可能没有)。用$help$可以对
这个对象的用法有一定程度的了解。\\
如果在一行后面加上警号$#$,Python解释器便不会去解释这一行警号后面的内容。
警号后面的内容叫做注释,一般用来在代码中做注解。
\begin{paperbox}{\textbf{Learning By Doing}\starfive}
运行以下代码,观察结果。
\begin{lstlisting}[style=PythonStyle1, caption=Rectangle]
import turtle
t1 = turtle.Turtle() # 创建一个小海龟对象
print(type(turtle)) # 输出模块turtle的类型
print(type(t1))
print(type(t1.fd))
print(type(50))
print(type(print))

print(turtle.__file__)
print(turtle.__doc__)

help(t1)
\end{lstlisting}
\end{paperbox}
